\documentclass[a4paper, 11pt, utf8]{article}
    \usepackage[utf8]{inputenc}
    \usepackage[english,russian]{babel}
    \usepackage[left=2cm,right=2cm,top=2cm,bottom=2cm,bindingoffset=0cm]{geometry}
    \frenchspacing
    

\author{А.\,П.\,Кузнецов, кафедра ФТИ ФФ НГУ, гр.\,7312}
\title{<<Инструменты построения и анализа структурного выравнивания молекул биополимеров для проекта UniPro UGENE>>\\\itshape Аннотация}
\date{}

\begin{document}

\maketitle
\thispagestyle{empty}

\paragraph{Введение.}
Одной из важнейших областей исследования в биологии, физиологии и фармацевтике является изучение биологических макромолекулярных структур, в частности белков и нуклеиновых кислот. В общем случае биологическая макромолекулярная структура представляет собой набор трехмерных координат образующих её атомов. Эти координаты могут быть получены экспериментально с помощью методов рентгеноскопии и атомной спектроскопии или предсказаны теоретически исходя из кодирующей последовательности с помощью вычислительных алгоритмов. Знание пространственной структуры биополимера необходимо, так как именно от неё в значительной степени зависит функциональное назначение биополимера и проявляемые им свойства. Механика взаимодействия макромолекул тоже во многом определяется их трехмерной структурой и взаимным расположением в пространстве.  

Необходимым средством анализа биологических макромолекулярных структур является инструмент построения структурного выравнивания, позволяющий сравнивать структуры между собой. Структурное выравнивание имеет важное значение при решении ряда задач структурной биологии:
\begin{itemize}
    \item поиск гомологии среди биологических объектов
    \item классификация биополимеров  по структурным особенностям
    \item определение функционального назначения биополимеров
    \item предсказание трехмерной структуры молекул биополимеров 
    \item и другие...
\end{itemize}

\paragraph{Цели.}
Целью работы является разработка инструмента построения структурного выравнивания и средств его визуализации. Важным условием является интеграция разработанного решения с проектом Unipro UGENE. UGENE -- это мощный инструмент автоматизации биологических исследований, с открытым исходным кодом, поддерживающий большое количество средств анализа последовательностей ДНК, РНК и аминокислот. Интеграция разработанных компонентов с уже существующим набором средств  UGENE позволит создать комплексный инструмент многоуровневого анализа биологических объектов. 

Сведения и навыки, полученные мной в этой области, послужат базисом для дальнейших разработок.

\end{document}

