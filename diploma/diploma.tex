\documentclass[a4paper, 14pt, titlepage, utf8]{extarticle}

\usepackage[utf8]{inputenc}
\usepackage[english,russian]{babel}

% TODO титульник и оформление
%	содержание
%	список литературы
%	спеллчек

\begin{document}

\section{Введение}	% название данной секции стандартно

Биоинформатика в наши дни делает значительные успехи. Позволяя решать многие задачи аналитически, не прибегая к сложным и дорогостоящим экспериментам. Одним из важных направлений является структурная биология. Предложено множество способов анализа структур. Одним из важнейших инструментов является сравнение структур.
Целью работы являестя разработка инструмента сравнения биологических структур. Работа выполняется на базе открытого проекта UGENE. (Тут телега о том какой юджин крутой). 
как вариант в юджине есть базовые вещи для работы со структурами, закономерным шагом является расширение инструментария, в качестве такого аналорасширителя выбрано структурное выравнивание.
(Тут о том что это все облегчит работу исследователя и улучшит качество его жизни -- сахар короче) 

\section{Описаие предметной области}	% название данной секции стандартно

\subsection{О структуре белка}
Третичная структура это трехмерная модель молекулы белка (или все же биополимера?). Структура молекулы играет важнейшую роль. Средний протеин содержит 300? остатков а значит всего 20\^300(это кстати больше чем атомов во вселенной) вариантов из которых в живых организмах присутствуют очень маленькое число. Вариантов сворачивания белка тоже до черта. Структурная биология важное направление в биологии. Она активно развивается. Сейчас эти структуры получают такими: NMR, X-Ray методами а так же моделируют (предсказывают). Сегодня выпускают по N (книга Structural Bioinformatics) в месяц и уже известно M? структур.

\subsection{О структурном выравнивании}
Структурное выравнивание -- это метод определения схожести структур по их третичной структуре.% зафиксировать уже это определение % 
Структурное выравнивание является важным инструментом имеет несколько важных применений:
	(взято из книги Structural Bioinformatics)
	- Классификация белков по группам, и создание библиотек паттернов для последующего аннотирования
	- Сравнение белка с известной функцией против неизвестного белка может помочь определить его функциональное назначение 
	- Методы предсказания структуры требуют сравнения предсказанных структур с заранее известными шаблонами
	- Структурное выравнивание может выявить связь между белками неявную из выравнивания последовательностей (там еще есть но пока непонятно)

Структурное выравнивание это NP полная задача, все существующие методы эвристические, кроме того разные методы могуть давать разные результаты.
Суть метода (тут картинку из презентации, не зря же я её корячился рисовал)

\subsection{Об алгоритмах}
Существует множество алгоритмов для решения данной задачи. (Тут надо как-то аккуратно рассказать о том что не все йогурты одинаково полезны, например есть алгоритмы тупо сравнивающие облака точек одинакого размера, это то любой школьник может !привет моя реализация! и их-то как грязи; а есть те которые стравнивают вообще ни разу непохожие структуры вот они то и крутые, есть варианты как в химере [и ссылочку на статью] которые строят на основе выравнивания последовательностей и тоже там какие-то результаты получают)
 
\section{Постановка задачи}	% название данной секции стандартно

Цель -- заиметь такой еба алгоритм в виде готового инструмента у себя в UGENE. 
Задачи.
Проанализировать готовые алгоритмы и выбрать подходящий с учетом этих -> (только C/C++, только открытый код, только хардкор) требований. Запилить абстрактный интерфейс для алгоритма структурного выравнивания. Засовать выбранынй алгоритм в плагин. Заделать гуевину чтобы даже полный идиот мог сделать выравнивание а не как во всех остальных тулах где только кодер который их разработал может это сделать. Сделать визуализацию построенного выравнивания на базе существующего BioStruct3DView. (вот вопрос: писать те задачи которые остались не сделанными или хитрожопо пропустить?) Сделать импорт и экспорт выравнивания в формат совместимый с другими инструментами. Покрыть все это дело тестами и вообще тщательно протестировать.

\section{Анализ существующих решений}

(Тут вставляестя простыня с тулами/библиотеками где в них нужно оыскать максимальное количество недостатков)

Потом надо подытожить недостатки и пообещать что наше решение ими страдать не будет.

\section{Анализ задачи}

Пока непонятно, вроде это уже написано выше в постановке.

\section{Реализация}

Тут про мучительный рефакторинг того что уже было BioStruct3DView \& BioStruct3D core.
Про адаптацию PTools к разным компиляторам например.
Про автоматические тесты чтоли написать.
Написать про subset и зачем он нужен. 
Про ёба абстрактный интерфейс который даже в core попал.
Про то что типы данных разные у нас и у PTools и что конвертер надо.

\section{Заключение и выводы}

Честно говоря это только перывй шаг и алгоритм выбран детский, но тем не менее он работает и в отличие от исходного вида им может пользоватсья каждый кто пожелает. Так же есть визуализация что не может не радовать. Так же этот алгоритм годится для обкатки API и всей обвязки (визуализация, импорт/экспорт). Ну и код у нас открытый велкам запиливать свои алгоритмы. Так же есть сообщество пользователей с которыми мы нацелены работать и пожелания которых всегда приоритетны. 

Так же я получил бесценный опыт промышленного шпионажа\^Wкодирования со всеми вытекающими и намерен развивать то что сваял дальше, благо, есть куда.

\section{Список литературы}

Книга Structural Bionformatics
Книга мат методы биоинформатики (надо найти, скачать, прочитать, мож че вставить даже)
Статья про Химеру?
Статья про PTools
Статья про UGENE?
Просмотреть список литературы у википедии

\end{document}
