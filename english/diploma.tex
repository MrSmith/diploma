\documentclass[a4paper, 12pt, titlepage, utf8]{extarticle}
    \usepackage[T2A]{fontenc}
    \usepackage[utf8]{inputenc}
    \usepackage[english]{babel}
    \usepackage[left=3cm,right=2cm,top=2cm,bottom=2cm,bindingoffset=0cm]{geometry}
    \usepackage{color}
    \usepackage{amsmath}
    \usepackage{cite}
    \usepackage[pdftex]{graphicx}
    \usepackage{subfig}
    \usepackage{numprint}
    \usepackage{amsthm} % для \pushQED
    \usepackage{comment}

\graphicspath{{./img/}{../img/diagrams/}{../img/}}
\frenchspacing

\linespread{1.3}                % полуторный интервал

\let\oldsection\section         % каждый раздел с новой страницы
\renewcommand{\section}{\newpage\oldsection}

% обёртка с моими настройками поверх figure:
% \begin{myfigure}{подпись}{fig:label} ... \end{myfigure}
\newenvironment{myfigure}[2]%
{\pushQED{\caption{#1} \label{#2}} % push caption & label
    \begin{figure}[h!tb]\centering } %
{  \popQED % pop caption & label
    \end{figure}}

% вставка картинки: \figure[params]{подпись}{file}
% создаёт label вида fig:file
\newcommand{\includefigure}[3][]{
\begin{myfigure}{#2}{fig:#3}
    \includegraphics[#1]{#3}
\end{myfigure}
}

% вставка subfigure внутри myfigure:
% \subfigure[params]{подпись}{file}
\newcommand{\subfigure}[3][]{
\subfloat[#2]{\label{fig:#3}\includegraphics[#1]{#3}}
}

\newcommand{\lookat}[1]{see fig.~\ref{#1}}

\bibliographystyle{gost780u}
%\bibliographystyle{unsrt}

\newcommand{\ruclass}{\textit}
\newcommand{\class}{\textbf}
\newcommand{\module}{\class}
\newcommand{\method}{\textit}

\addto\captionsenglish{
    %\renewcommand\contentsname{Содержание}
    % перекрываю \refname, чтобы список литературы сам добавлял себя в оглавление
    \let\oldrefname\refname
    \renewcommand\refname{\addcontentsline{toc}{section}{\oldrefname}\oldrefname}
}

\newcommand{\todo}[1]{\textbf{\textcolor{red}{TODO: #1}}}

\author{Alexey Kuznetsov, group 7362}
\title{Tools for structural alignment building and anlysis for UniPro UGENE}

\specialcomment{original}{%
    \begingroup
    \itshape
    \color{blue}
    %\Russian
}{
\endgroup
}
\excludecomment{original} % скрывает текст внутри окружения original

\begin{document}

% -------------------- Титульная страница------------------------------
\begin{titlepage}
\newpage

\thispagestyle{empty}
\begin {center}
МИНОБРНАУКИ РОССИИ

\vspace{0.3cm}

Государственное образовательное учреждение\\
высшего профессионального образования\\
«Новосибирский государственный университет» (НГУ)

\vspace{0.6cm}

Физический факультет

\vspace {2cm}

Квалификационная работа на соискание\\
степени бакалавра

\vspace {0.5cm}

Кафедра физико-технических исследований

\vspace {1cm}

Кузнецов Алексей Павлович

\vspace {1.5cm}

\textsc{\textbf{Инструменты построения и анализа структурного выравнивания молекул биополимеров для проекта UniPro UGENE}}

\vspace {1.5cm}

\begin{flushright}

Научный руководитель:

Фурсов Михаил Юрьевич,\\
ОАО <<Унипро>>

\end{flushright}

\vspace {3cm}

Новосибирск~--- 2011~год
\end{center}

\end{titlepage}



% -------------------- Содержание--------------------------------------
\tableofcontents
\newpage

% -------------------- Введение ---------------------------------------
\section{Introduction}      % название данной секции стандартно
\begin{original}
Исследования в области биоинформатики приобретают высокую актуальность в наши
дни.
Современные вычислительные мощности позволяют обрабатывать большие объемы данных
характерные для биологических объектов, и выполнять сложные математические
расчеты за приемлемое время. Биоинформатика позволяет исследователям эффективно
автоматизировать обработку данных, и в ряде случаев решать поставленные задачи
аналитически, не прибегая к дорогостоящим и длительным экспериментам.
\end{original}

Bioinformatics researches are highly relevant today.
Modern computational power allows to deal with large volumes of data typical for
biological objects and perform complex mathematical calculations in a
reasonable time. Bioinformatics tools allow researchers to efficiently automate
the processing of data and in some cases to solve problems analytically
without carrying out costly and time-consuming experiments.

\begin{original}
Одной из важнейших областей исследования в биологии, физиологии и фармацевтике
является изучение биологических макромолекулярных структур, в частности белков и
нуклеиновых кислот. Свойства и назначение этих соединений во многом определяются
их структурой. Поэтому исследование структуры биополимеров позволяет лучше
понимать процессы, протекающие в живом организме, находить связи между
биологическими объектами и выяснять причины возникновения возможных патологий и
заболеваний а значит создавать новые высокоэффективные препараты. Инструменты
анализа созданные с помощью вычислительной биологии способствуют достижению этих
целей и являются неотъемлемой частью современной науки. Существует множество
методов автоматизации анализа структур таких как трехмерная визуализация,
структурное выравнивание, предсказание вторичной и третичной структур белка из
кодирующей последовательности, моделирование функциональности и взаимодействия
биополимеров и другие.
\end{original}

One of the major fields of research in biology, physiology and pharmaceutics is
the study of biological macromolecular structures, in particular proteins and
nucleic acids. Properties and function of these compounds are largely determined
by their structure. Therefore, investigation of the structure of biopolymers can
help to understand the processes occurring in living organisms, to find
relations between biological objects and investigate the possible causes of
pathologies and diseases and thus create new high-performance medicine. Analysis
tools created by the computational biology contribute to the achievement of
these goals and are an integral part of modern science. There are many methods
for automating the analysis of structures such as three-dimensional
visualization, structural alignment, the prediction of secondary and tertiary
structures based on the protein coding sequence, the biopolymers function
simulation and others.

\begin{original}
Необходимым средством анализа при решении ряда задач структурной биологии, таких
как: поиск гомологии среди объектов, классификация по структурным особенностям,
определение функционального назначения и предсказание структуры биополимеров,
является инструмент построения структурного выравнивания, позволяющий сравнивать
структуры между собой.
\end{original}

An essential tool for the structure analysis is a tool for building structural
alignment allowing the structure to compare with each other. It is necessary
for solving a number of structural biology problems such as: search for homology
among the objects, the classification of structural features, the definition of
functionality and structure prediction of biopolymers,

\begin{original}
Целью данной работы является разработка инструмента построения структурного
выравнивания и средств его визуализации. Работа выполняется базе проекта с
открытым исходным кодом Unipro UGENE \cite{ugene}. UGENE -- это комплексный
инструмент автоматизации биологических исследований, поддерживающий большое
количество средств анализа последовательностей ДНК, РНК и аминокислот. UGENE
является мультиплатформенным приложением, позволяющим достичь максимальной
производительности работы за счет платформо-зависимых оптимизаций и адаптации
алгоритмов для выполнения в многопоточных средах. В настоящий момент в UGENE уже
реализованы средства визуализации трехмерных структур, а также некоторые базовые
методы структурного анализа, таким образом, реализация инструмента построения
структурного выравнивания является закономерным шагом в развитии проекта.
\end{original}

The aim of this work is the development of a tool for building and visualizing
structural alignment which is based on the open source project Unipro UGENE
\cite {ugene}. UGENE is a comprehensive tool for automating biological research,
which supports a large number of tools for analyzing sequences of DNA, RNA and
amino acids. UGENE is a multiplatform application that allows to achieve
maximum productivity through platform-specific optimizations. At the moment
UEGENE is capable of visualizing three-dimensional structures as well as some
of the basic methods of structural analysis, therefore the implementation tools
of structural alignment is a natural step in the evolution of the project.

\begin{original}
Интеграция разработанных компонентов с уже существующим набором средств UGENE
позволит создать комплексный инструмент многоуровневого анализа биологических
объектов, который даст возможность эффективней решать сложные многоэтапные
задачи, связанные с макромолекулярными структурами, что несомненно повысит
результативность работы исследователей в этой области.
\end{original}

The integration of components developed with an existing set of UGENE tools will
bring a comprehensive tool for multi-level analysis of biological objects. This
tool will enable more effective solutions for complex multi-step tasks
associated with macromolecular structures and will undoubtedly increase the
efficiency of researchers in this field.

% В чем научная новизна?


% -------------------- Описание предметной области --------------------
\section{Overview of the subject area}	% название данной секции стандартно
\begin{original}
Анализ макромолекулярных структур это обширная область исследований в
современной биологии. Она охватывает большой спектр объектов и явлений и
располагает множеством инструментов вычислительной биологии. Ниже речь пойдет о
структурном выравнивании, одном из базовых методов анализа структур. Также будет
конкретизирован термин <<биологическая макромолекулярная структура>> и
рассмотрена база данных PDB, предназначенная для хранения описаний структур.
\end{original}

The analysis of macromolecular structures is a vast area of research in modern
biology. It covers a wide range of objects and phenomena and offers many tools
for computational biology. Here we will focus on the structural alignment, one
of the basic methods of analysis of structures.

\begin{original}
В рамках этой работы, термин <<биологическая макромолекулярная структура>>
рассматривается применительно к белкам, тем не менее, некоторые методы
структурного анализа, например, структурное выравнивание, возможно обобщить и
для нуклеиновых кислот (ДНК и РНК).
\end{original}

At first the term "biological macromolecular structure" will be defined. Note
that this term is applied to the proteins, however some methods of structural
analysis, for example structural alignment may be generalized to the nucleic
acids (DNA and RNA).

\subsection{Definition of macromolecular biostructure}
\begin{original}
В общем случае биологическая макромолекулярная структура представляет собой
набор трехмерных координат образующих её атомов (\lookat{fig:biostructure}).
Эти координаты могут быть получены экспериментально с помощью методов
рентгеноскопии и атомной спектроскопии или предсказаны теоретически исходя из
кодирующей последовательности с помощью вычислительных алгоритмов. Знание
пространственной структуры биополимера необходимо, так как именно от неё в
значительной степени зависит функциональное назначение биополимера и проявляемые
им свойства. Механика взаимодействия макромолекул тоже во многом определяется их
трехмерной структурой и взаимным расположением в пространстве.

%\includefigure[width=0.8\linewidth]{Условное изображение молекулы белка (1GNA)}{biostructure}
\end{original}

In general, the biological macromolecular structure is a set of
three-dimensional coordinates of atoms (\lookat{fig:biostructure}). These
coordinates can be obtained experimentally using the methods of X-ray and atomic
spectroscopy, or theoretically predicted on the basis of the coding sequence by
using computational algorithms. Knowledge of the spatial structure of the
biopolymer is necessary because its it functionality and properties are largely
dependent on it. The mechanics of the interaction of macromolecules is also
largely determined by their three-dimensional structure and the relative
positions in space.


\includefigure[width=0.8\linewidth]{Conventional representation of protein
molecule (1GNA)}{biostructure}

\begin{original}
Однако, координаты атомов далеко не единственная составляющая макромолекулярной
структуры. Принято выделять следующие уровни организации биологической
структуры:
\begin{itemize}
    \item \textbf{Первичная структура} или кодирующая последовательность.
Первичная структура эквивалентна химической формуле биополимера.
    \item \textbf{Вторичная структура} -- локальные упорядочивания фрагментов
главной цепи макромолекулы, стабилизированные преимущественно водородными
связями. Вторичная структура полностью задается кодирующей последовательностью.
    \item \textbf{Третичная структура} -- пространственное строение
макромолекулы (набор пространственных координат составляющих её атомов). Зависит
не только от первичной и вторичной структур но и в больш\'{о}й степени от
окружающей среды в которой биополимер синтезирован.
    \item \textbf{Четвертичная структура} или доменная представляет собой
взаимное расположение нескольких макромолекул в составе единого комплекса
\end{itemize}
\end{original}

However, the coordinates of the atoms is not the only component of the
macromolecular structure. It is common to distinguish the following levels of
biological organization structure:
\begin{itemize}
    \item \textbf{Primary structure} or the coding sequence. The primary
structure is equivalent to the chemical formula of the biopolymer.
    \item \textbf{Secondary structure} is a set of a local sequencing fragments
of the macromolecule backbone, primarily stabilized by hydrogen bonds. Secondary
structure is completely determined by the coding sequence.
    \item \textbf{Tertiary structure} is a spatial structure of the
macromolecule (a set of spatial coordinates of its constituent atoms). Depends
not only on the primary and secondary structures but also on the environment in
which the biopolymer was synthesized.
    \item \textbf{Quaternary structure} or domain structure is the relative
position of several macromolecules entering in the composition of the a complex.
\end{itemize}

\begin{original}
Таким образом, биологическая макромолекулярная структура может включать в себя:
информацию о кодирующей последовательности, информацию о вторичной структуре и
доменах (функциональных областях молекулы), информацию о пространственном
строении молекулы, кроме того в понятие структуры включают полный граф
химических связей. Дело в том, что граф химических связей, как правило,
невозможно восстановить из кодирующей последовательности, из-за т.н.
посттрансляционных модификаций. Под воздействием окружающей среды структура и
химический состав биополимера могут значительно меняться, в состав белков могут
войти нестандартные аминокислоты или другие соединения. Для сохранения этой
информации и служит граф химических связей, явно описывающей все соединения
между атомами.
\end{original}

Thus, the biological macromolecular structure may include: information on the
coding sequence, information about the secondary structure and domains
(functional areas of the molecule), information on the spatial structure of the
molecule. In addition it may include a complete graph of chemical bonds, usually
it is impossible to recover it from the coding sequence, because of the
so-called posttranslational modifications. Under the influence of environment
structure and chemical composition of the biopolymer can vary considerably, the
proteins may include non-standard amino acids or other compounds, this
information also saved in graph of chemical bonds, clearly describing the
connections between atoms.

\begin{original}
Дополнительно макромолекулярная структура может содержать различную мета
информацию: сведения о том, как структура была получена, описание
кристаллографической ячейки, замечания о функциональных особенностях молекулы,
ссылки на связанные научные статьи и так далее.
\end{original}

In addition, macromolecular structure may contain a variety of meta-information:
information about how the structure was obtained, a description of the
crystallographic cell, remarks on the functional characteristics of the
molecule, links to related research papers and so on.

\subsection{PDB database}
\begin{original}
В настоящее время существует несколько общедоступных баз данных, аккумулирующих
информацию об известных биологических структурах. Среди них можно выделить
Protein Data Base \cite{pdb}, в ней доступны все известные на сегодняшний день
структуры (более 70000,  изначально в 1971 году в ней было доступно всего 7
структур). Лаборатории, специализирующиеся на изучении биологических структур,
пополняют в первую очередь эту базу данных.
\end{original}

Currently, there are several publicly available databases accumulating
information on known biological structures. Among them are the Protein Data Base
\cite{pdb}, all known to the date structures available there (more than 70000,
originally in 1971 it was available only 7 structures). Laboratories
specializing in the study of biological structures fill up first this database.

\begin{original}
Данные PDB хранятся в простом текстовом формате. Этот формат получил широкое
распространение и является стандартом для инструментов анализа структур, он
хорошо подходит для записи результатов структурного выравнивания, о котором
пойдет речь ниже.
\end{original}

PDB data is stored in plain text. This format is widely spread and is a standard
for structure analysis tools. It is also well suited for recording the results
of structural alignment, which will be discussed below.

\subsection{Structural alignment}
\begin{original}
Одним из важных направлений анализа трехмерных структур является поиск сходства
среди макромолекул. Эта проблема -- прямая аналогия с очень распространенной в
биоинформатике задачей поиска схожих последовательностей. Отличие в том, что
поиск основан на нахождении общих черт трехмерных моделей. Иными словами
структурное выравнивание -- это способ определения степени схожести молекул по
их трехмерной структуре.

\includefigure[clip, trim=0 19.5cm 0 1cm, width=\linewidth]{Структурное
выравнивание}{method.pdf}
\end{original}

One of the important fields of analysis of three-dimensional structures is
finding similarities among the macromolecules. This problem is a direct analogy
with the very common task in bioinformatics -- searching for the similar
sequences. The difference is that the search is based on finding common features
of three-dimensional models. In other words, the structural alignment is a
method of determining the degree of molecule similarity based on their
three-dimensional molecular structure.

\includefigure[clip, trim=0 19.5cm 0 1cm, width=\linewidth]{Structural
alignment}{method.pdf}

\begin{original}
Задача построения структурного выравнивания в простейшем случае сводится к
поиску трансформации \textbf{T}, приводящей к оптимальному наложению. Качество
наложения задается метрикой
\begin{center} $ \operatorname{RMSD}(ref, mob) = \sqrt{ \dfrac{\sum_{i=1}^n \|
\mathbf{\bar{x}}_{ref,i} -  \mathbf{\bar{x}}_{mob,i} \| ^2} {n} } $ \end{center}
чем меньше \textbf{RMSD}, тем больше сходство между структурами,
\textbf{RMSD=0}  означает что структуры идентичны.

Как правило, одна структура остается неподвижной (reference) и трансформация
вычисляется только для второй, т.н. мобильной (mobile) структуры.
\end{original}

In simplest case the problem of building structural alignment is reduced to
searching  a transformation \textbf{T} that will give an optimal superposition.
Quality of superposition can be measured as
\begin{center} $ \operatorname{RMSD}(ref, mob) = \sqrt{ \dfrac{\sum_{i=1}^n \|
\mathbf{\bar{x}}_{ref,i} -  \mathbf{\bar{x}}_{mob,i} \| ^2} {n} } $ \end{center}
less \textbf{RMSD} means better similarity between structures, \textbf {RMSD =
0} means that the structures are identical.

Typically one structure is fixed (reference) and the transformation is computed
only for a second so-called mobile structure.

\begin{original}
Структурное выравнивание является важным инструментом анализа и имеет ряд
применений \cite{structural-bionformatics}.
\begin{itemize}
    \item Классификация биополимеров по структурным паттернам и составление
библиотек паттернов для последующего аннотирования
    \item Определение функционального назначения белка путем сравнения его
структуры со структурой известного белка
    \item Структурное выравнивание может обнаруживать связь между биополимерами,
неявную исходя из выравнивания последовательностей
    \item Методы предсказания структуры для оценки качества результатов требуют
сравнения смоделированных структур с заранее известными шаблонами
    \item Структурное выравнивание необходимо как вспомогательный инструмент при
конструировании искусственных молекул
\end{itemize}
\end{original}

Structural alignment is an important analysis tool and has a number of
applications \cite{structural-bionformatics}.
\begin{itemize}
    \item Biopolymer classification based on a structural patterns and building
pattern libraries for further annotaiton
    \item Defining functionality of the protein by comparing its structure with
the structure of a known protein
    \item Structural alignment can detect the connection between biopolymers
which can be implicit from the sequence alignment
    \item Structure prediction methods for assessing the quality of the results
require a comparison of the simulated structures against known patterns
    \item  Structural alignment is necessary as an auxiliary tool for the design
of artificial molecules
\end{itemize}

\paragraph{Input and output.}
% вход
\begin{original}
Алгоритм парного структурного выравнивания на входе принимает две трехмерные
структуры. Структура может представлять как всю макромолекулу или являться
некоторым участком её цепи. Некоторые методы позволяют выравнивать только
участки цепи одинаковой длинны. Выбор оптимальных участков в таком случае
становится отдельной задачей, например, он может быть сделан на основе
выравнивания последовательностей. Существуют также методы предназначенные только
для сравнения структур с идентичными последовательностями, они в первую очередь
полезны для сравнения разных конформаций одного и того же биополимера.
\end{original}

Pair structural alignment algorithm accepts two three-dimensional structures.
The structure can be represented as a whole macromolecule, or be a certain
portion of the chain. Some methods allow you to align only the parts of a chain
of equal length. Choosing the best sites in this case is a separate problem, for
instance decision can be based on sequence alignments. There are also techniques
intended only for the comparison of structures with identical sequences, they
are primarily useful for comparing different conformations of the same
biopolymer.

\begin{original}
Известно, что молекулы белков представляют собой линейные полимеры, при этом
можно выделить основную цепь молекулы и присоединенные к ней боковые цепи
(радикалы аминокислот) \lookat{fig:backbone}. Существуют различные варианты для
выбора подмножества атомов которые будут участвовать в выравнивании и расчете
RMSD. При выравнивании структур с сильно различающимися последовательностями,
атомы боковых цепей не учитываются. Иначе при расчете RMSD большое число не
совпадающих атомов боковых цепей будет преобладать, даже в случае если основные
цепи совпадут. По этой причине многие алгоритмы структурного выравнивания
по-умолчанию учитывают только атомы основной цепи. Дополнительно, для упрощения
вычислений могут учитываться только альфа-атомы углерода, так как координаты
альфа-атомов углерода полностью задают структуру основной цепи. Только в случае
когда кодирующие последовательности белков обладают высоким сходством или
полностью идентичны имеет смысл строить выравнивание с учетом атомов боковых
цепей, в таком
случае RMSD показывает не только сходство структуры основной цепи но и
совпадение положений атомов боковых цепей. Для улучшения результатов
выравнивания можно дополнительно учитывать совпадения вторичной структуры, схем
водородных, ковалентных и ионных связей и другие факторы \cite{wiki-3}.
\end{original}

It is known that protein molecules are linear polymers that can be identified
with the main chain of the molecule and attached to her side chains (amino acid
radicals) \lookat{fig:backbone}. There are various options for selecting a
subset of atoms that participate in the calculation of alignment and RMSD. For
aligning structures with very different sequences the atoms of the side chains
often are not counted. Otherwise the resulting RMSD would be biased by a
prevailing number of side chains atoms even if the main chains match. For this
reason, many structural alignment algorithms by the default account only the
backbone atoms. In addition to simplify the calculations only the alpha-carbon
atoms can be taken into account since the coordinates of the alpha-carbon atoms
completely determine the structure of the main chain. Only in the case when the
protein coding sequences have high similarity or even are identical it makes
sense to build the alignment including atoms of the side chains, then  RMSD
shows not
only the similarity of the structure of the main chain but also the coincidence
of positions of the atoms of side chains. The coincidence of the secondary
structure diagrams, hydrogen, covalent and ionic bonds, and other factors can be
concidered for improving the quiality of alignment \cite {wiki-3}.

\begin{original}
\includefigure[width=0.9\linewidth]{Участок основной цепи молекулы белка.
Боковые цепи изображены тонкими линиями. Цветами обозначены химические элементы:
углерод(C) -- зеленым, азот(N)~--~синим, кислород(O) -- красным.}{backbone}
\end{original}

\includefigure[width=0.9\linewidth]{Protein backbone region. Side chains are
shown in thick lines. Chemical elements shown with coolr: carbon(C) -- green,
nitrogen(N)~--~blue, oxygen(O) -- red.}{backbone}

% выход
\begin{original}
Результатом работы алгоритма или структурным выравниванием являются
преобразованные трехмерные структуры и вычисленное значение RMSD. Вместо
преобразованных координат мобильной структуры достаточно рассматривать только
матрицу (4$\times$4) задающую преобразование сдвига/поворота. Формат PDB
позволяет записать обе структуры, reference и mobile, в один файл, что делает
его удобным для хранения структурного выравнивания.
\end{original}

The results of the algorithm or a structural alignment are converted
three-dimensional structures and the calculated value of RMSD. It is reasonable
to consider only the matrix (4$\times$4) which specifies the transformation of
the shift/rotation instead of whole set of transformed coordinates. PDB format
allows to store both reference and mobile structures  in a single file making it
suitable for storing the structural alignments.

\newpage
\paragraph{Relation with sequence alignment.}
\begin{original}
Наряду с методами, опирающихся исключительно на информацию о трехмерной
структуре, существуют подходы учитывающие также информацию о кодирующей
последовательности. Первым шагом работы таких алгоритмов является построение
выравнивания последовательностей. Наличие такого выравнивания позволяет выделить
участки молекулярных цепей, где последовательности похожи или идентичны, и
рассматривать в дальнейшем только эти участки при сравнении структур. Имеет
место и обратная задача -- построение выравнивания последовательностей по
известному наложению структур.
\end{original}

In addition to methods that rely solely three-dimensional structure there are
approaches that  taking into account information of the coding sequence. The
first step of these algorithms is the construction of sequence alignment. The
presence of such alignment can identify areas of molecular chains where the
sequences are similar or identical and consider further only those sections when
comparing structures. There is also the inverse problem -- the construction of
sequence alignment based on known structural superposition.

\paragraph{Algorithmic complexity.}
\begin{original}
Не смотря на то, что некоторые близкие проблемы структурной биоинформатики
являются NP-полными, для задачи структурного выравнивания NP-полнота не
доказана. Более того, точное решение известно для некоторых специальных метрик
GDT\_TS \cite{wiki-2} и MaxSub \cite{wiki-11}. Но практическое применение точных
алгоритмов использующих эти метрики не целесообразно из-за все еще большой
вычислительной сложности, которая, зависит не только от размера входных
структур, но и от их строения  \cite{wiki-12}. Большинство методов полагаются на
эвристические алгоритмы.
\end{original}

Despite the fact that some of the close structural bioinformatics problems are
NP-complete  the problem of structural alignment isn't proven to be NP-complete.
Moreover the exact solution is known for certain special metrics GDT\_TS
\cite{wiki-2} and MaxSub \cite{wiki-11}. But the real application of exact
algorithms using these metrics is impractical due to the still high
computational complexity which depends not only on the size of the input
structures but also on their structure features \cite{wiki-12}. Most methods
still rely on heuristic algorithms.

\paragraph{Structural alignment visualization.}
\begin{original}
Трехмерная визуализация является естественным способом представления
структурного выравнивания. Она преследует те же цели что и трехмерная
визуализация макромолекулярных структур. Прежде всего, она позволяет нагляднее
представлять громоздкие химические формулы соответствующие сложным соединениям.
Визуализация помогает увидеть новые особенности исследуемых структур или
закономерности в строении биополимера. Еще одним важным применением визуализации
является использование изображений в научных публикациях и образовательных
ресурсах с целью демонстрации свойств биологического объекта или в виде
пояснения для идей и выводов.
\end{original}

Three-dimensional visualization is a natural way to represent the structural
alignment. It pursues the same objectives as the three-dimensional visualization
of macromolecular structures. First it allows you to graphically represent the
bulky chemical formula corresponding to complex compounds. Visualization helps
to see the new features of the structures or patterns in the structure of the
biopolymer. Another important application is the use of image visualization in
scientific publications and educational resources in order to demonstrate the
properties of a biological object or in the form of ideas and explanations for
the findings.


% -------------------- Обзор существующих решений----------------------
\section{Overview of existing solutions}
\begin{original}
На данный момент существует широкий ряд программных продуктов позволяющих решать
различные задачи в области автоматизации анализа и визуализации биологических
макромолекулярных структур. Разработано большое количество алгоритмов
выравнивания структур реализующих различные подходы. Многие из этих реализаций
представлены только в виде веб-сервисов. Значительным недостатком таких решений
является большое время выполнения анализа, это создает определенные неудобства и
делает невозможным многократный запуск выравниваний. Гораздо меньшее количество
программ позволяют выполнять выравнивание локально и использовать этот
инструмент в совокупности с другими видами анализа. В этом разделе кратко
рассмотрены наиболее эффективные и широко используемые из таких решений.
\begin{original}

At the moment there is a wide range of software which is intended for solving
various problems in the automation of analysis and visualization of biological
macromolecular structures. There is a number of algorithms for aligning
structures implementing different approaches. Many of these implementations are
presented only as a Web service. A significant drawback of such solutions is the
large analysis run-time, it creates some inconvenience and makes it impossible
to run multiple alignments. A much smaller number of programs allows to preform
alignment locally and use this tool in conjunction with other types of analysis.
In this section we briefly discuss the most effective and widely used of these
solutions.

% --- The Swiss Institute of Bioinformatics Swiss-PdbViewer (Deep View) ---
\begin{original}
\paragraph{Swiss-PdbViewer (Deep View) \cite{deep-view}}
приложение с дружественным графическим интерфейсом. Включающее широкий спектр
средств анализа и визуализации биологических структур. В том числе и несколько
инструментов сравнения структур:
\begin{itemize}
    \item \textbf{Magic Fit} производит наложение структур на основе
выравнивания последовательностей.
    \item \textbf{Iterative Magic Fit} производит наложение структур с помощью
\textbf{Magic Fit}, и затем выполняет оптимизацию RMSD.
    \item \textbf{Explore Domain Alternate Fits} вычисляет структурное
выравнивание учитывая при этом только третичную структуру.
\end{itemize}
\end{original}

\paragraph{Swiss-PdbViewer (Deep View) \cite{deep-view}}
is an application with a user-friendly graphical interface. It includes a
wide range of biological structures analysis and visualization tools, including
a number of structure comparison tools:
\begin{itemize}
    \item \textbf{Magic Fit} produces a structure superposition based on
sequence alignment.
    \item \textbf{Iterative Magic Fit} produces a structure superposition using
\textbf{Magic Fit}, and then performs the RMSD optimization.
    \item \textbf{Explore Domain Alternate Fits} computes a structural alignment
based only on the tertiary structure.
\end{itemize}

\begin{original}
Среди особенностей можно отметить возможность выравнивать с помощью инструментов
Magic Fit и Iterative Magic Fit модели состоящие из нескольких цепей и
визуализацию с применением специальной цветовой схемы, где интенсивность цвета
зависит от величины расхождения молекулярных цепочек.

Недостатком является отсутствие поддержки Linux платформ.
\end{original}

The features include an opportunity to align with the Magic Fit and Iterative
Magic Fit tools  models consisting of several chains, and also visualization
using a special color scheme, where the intensity of the color depends on the
divergence of the molecular chains.

The lack of support for Linux platforms is considered as main disadvantage.

% --- UCSF Chimera  ---
\begin{original}
\paragraph{UCSF Chimera \cite{chimera}}
кроссплатформенный инструмент визуализации биологических структур, реализованный
на языке Python. Наряду с визуализацией предоставляет некоторые возможности
анализа структур. Построение структурного выравнивания производится в два этапа:
сначала производится выравнивание последовательностей, а затем на основе него
выполняется наложение трехмерных структур \cite{chimera-alignment}. При этом,
могут учитываться особенности вторичной структуры. Есть возможность решать
обратную задачу: вычислять выравнивание последовательностей по существующему
выравниванию структур.
\end{original}

\paragraph{UCSF Chimera \cite{chimera}}
is a cross platform tool intended for biostructure visualization written in Python.
As well as visualization capabilities is contains ability to compare structures.
Structure alignment is divided into two steps: first is the sequence alignment,
and the second is a based on it structure imposition \cite{chimera-alignment}.
Secondary structure features are taken into account. Reverse problem of aligning
sequences based on structural superposition can also be solved.

% --- PyMOL ---
\begin{original}
\paragraph{PyMOL \cite{pymol}}
программа на языке Python, предназначенная для визуализации биологических
структур, с возможностью строить наложение трехмерных структур через
выравнивание последовательностей. При визуализации выравнивания может быть
использована цветовая схема, в которой цветом обозначена величина расхождения
молекулярных цепочек. Несмотря на то, что программа имеет графический интерфейс,
выполнения выравнивания возможно только через текстовую консоль, или посредством
написания скрипта.
\end{original}

\paragraph{PyMOL \cite{pymol}}
is a Python program mostly intended for biostructure visualization with capability of
building structural alignment based on sequence alignment. Advanced structural alignment
visualization features include possibility to apply color scheme which stress structure
differences. In spite of having graphical interface the only possibility to build
structural alignment is writing Python script.

\begin{original}
\paragraph{}
Веб-сервисы построения структурного выравнивания \textbf{FAST} и
\textbf{MultiProt} предоставляют для загрузки дистрибутивы программ, которые
могут быть использованы для построения выравнивания на локальной машине. Эти
программы предназначены только для структурного выравнивания и требуют
использования сторонних пакетов для визуализации. Для таких решений характерно
отсутствие поддержки операционной системы Microsoft Windows.
\end{original}

\paragraph{}
\textbf{FAST} and \textbf{MultiProt} web-services also provide standalone tools
for structural alignment. These packages are highly specialized in building
alignments and require third-party tools for visualization. Lack of Windows
platforms support is typical for these solutions.


% -------------------- Постановка задачи ------------------------------
\section{Problem statement}	% название данной секции стандартно

\subsection{Analysis of existing solutions}
\begin{original}
Вышеперечисленные решения в области анализа и визуализации биологических
макромолекулярных структур (см. раздел 3) имеют ряд недостатков. Условно можно
разделить все решения на две категории: специализированные на построении
структурного выравнивания, и универсальные инструменты структурного анализа, в
которых имеется подобная функциональность. Для первых характерны отсутствие
интеграции с другими инструментами и, как следствие, невозможность протяжки
результатов алгоритма для дальнейшего анализа, для вторых -- концентрация
внимания только на трехмерной модели и слабая связь с биологической
последовательностью, являющейся основой структуры. К общим недостаткам можно
отнести неудобный визуальный пользовательский интерфейс или его отсутствие,
невысокая производительность решений, реализованных на интерпретируемых языках и
ограниченная поддержка платформ у нативных решений.
\end{original}

The described in the sec. 3 biological macromolecular structures analysis and
visualization solutions have number of disadvantages which can limit their
usage. For convenience these tools can be divided into two categories:
specialized in building structural alignment and universal toolkits which
include structural analysis as well as lots of other instruments. For the first
group is typical the absence of integration with other tools and as a consequence
the impossibility of usage outcomes of the algorithm for further analysis. For
the second focusing only on the three-dimensional model and a weak relationship
with the biological sequence  which is the basis of the structure. Common
disadvantages can include an inconvenient graphical user interface or low
performance for solutions implemented in an interpreted languages and limited
support for platforms for native solutions.

\begin{original}
Реализация инструмента построения выравнивания биологических структур на базе
платформы Unipro UGENE позволит преодолеть эти недостатки.
\end{original}

The implementation of structural alignment tools on the basis of Unipro UGENE
platform can help to overcome these limitations.

\newpage
\subsection{Objectives}
\begin{original}
Целью дипломной работы является создание программного компонента, позволяющего
автоматизировать и упростить процесс построения структурного выравнивания
биологических макромолекулярных структур и интеграция этого инструментария с
системой Unipro UGENE.
\end{original}

The goal of the work is to create a set of software components that allow
automate and simplify the process of constructing a structural alignment of
biological macromolecular structures. These tools should be integrated into
Unipro UGENE bioinformatics platform.

\begin{original}
Созданный набор компонентов должен обеспечить:
\begin{itemize}
    \item \textbf{Инструмент построения структурного выравнивания} на базе
актуального алгоритма.

    \item \textbf{Графический интерфейс пользователя.} Дружественный графический
интерфейс должен способствовать эффективному решению задач стоящих перед
исследователем.

    \item \textbf{Интеграцию} компонента с существующими инструментами анализа
UGENE.

    \item \textbf{Трехмерную визуализацию структурного выравнивания.}
Существующий модуль \module{BioStruct3DView} должен быть доработан для
отображения выравниваний. Необходимо иметь возможность выводить одновременно
несколько структур, гибко управлять параметрами отображения молекул. Необходимо
добавить специальную цветовую схему, которая сделает визуализацию более
наглядной.

    \item \textbf{Переносимость и производительность.} Под переносимостью
понимается возможность работы программы на широком спектре различных платформ.
Производительность набора программных модулей должна быть не ниже чем
производительность существующих решений и реализаций методов анализа.

    \item \textbf{Базу для будущего развития} и добавления новых возможностей.
Сюда входит добавление новых алгоритмов выравнивания, улучшение визуализации,
интеграцию с внешними инструментами путем реализации импорта и экспорта
выравниваний и другие задачи.

    \item \textbf{Тестовую базу}, которая гарантирует корректность работы
разработанного компонента.
\end{itemize}
\end{original}

Designed toolkit should provide following capabilities:
\begin{itemize}
    \item \textbf{Structural alignment tool} based on an actual algorithm.

    \item \textbf{Graphical user interface.} User-friendly graphical interface
        must effectively address the challenges facing researcher-biologist.

    \item \textbf{Tight integration} with the existing UGENE toolkit.

    \item \textbf{3D visualization of alignment.}
        Existing visualization module \module{BioStruct3DView} should be enhanced
        for displaying alignments. User should be able view both or several structures,
        flexibly control the parameters of the mapping of the molecules. Particular
        color scheme is necessary to make visualization more descriptive.

    \item \textbf{Portability and performance.} Portability refers to ability to
        run on variety of platforms supported by UGENE framework. At the same time the
        solution must provide competitive performance comparable with existing
        widespread solutions.

    \item \textbf{Extensibility.} This includes the ability of further integration of
        new alignment algorithms for alignment, extensible visualization engine and
        integration with external tools through the implementation of alignments import
        and export.

    \item \textbf{Test base}, which will validate functionality correctness.
\end{itemize}

% -------------------- Реализация -------------------------------------
\begin{original}
\section{Реализация}
\subsection{Внутреннее представление данных UGENE}
В UGENE существует модель данных для представления биологической
макромолекулярной структуры \class{BioStruct3D} (\lookat{fig:BioStruct3D}). Эта
модель разработана так, чтобы включать все данные о молекулярной структуре
необходимые для её анализа.

%%\includefigure[width=\linewidth]{Модель данных UGENE \class{BioStruct3D}}{BioStruct3D}
\includefigure[scale=0.6]{Модель данных UGENE \class{BioStruct3D}}{BioStruct3D}

В модель данных макромолекулярной структуры (\class{BioStruct3D}) входят
следующие базовые компоненты:
\begin{itemize}
    \item \textbf{pdbID} идентификатор биополимера в базе данных PDB
    \item \textbf{AtomSetMap} набор коллекций описаний атомов, составляющих
структуру
    \item \textbf{MoleculeMap} набор графов химических связей биополимера
    \item \textbf{SecondaryStructureMap} набор описаний вторичной структуры
\end{itemize}

Инструментам анализа UGENE не требуются сведения о том, каким образом создана
структура: загружена ли она из файла, базы данных, или смоделирована в ходе
работы одного из методов анализа структуры. Модель не зависит от какого-либо
конкретного файлового формата  благодаря абстракциям \class{Document} и
\class{DocumentFormat} (\lookat{fig:DocumentModel}). \ruclass{Документ}
предоставляет доступ к некоторому набору объектов, которыми оперирует система.
Такими объектами являются, например, биологическая структура
(\class{BioStruct3DObject}) или биологическая последовательность
(\class{SequenceObject}).

%%\includefigure[width=\linewidth]{Документ}{DocumentModel}
\includefigure[scale=0.6]{Документ}{DocumentModel}

\newpage
Конкретные реализации интерфейса \class{DocumentFormat} отвечают за поддержку
соответствующих файловых форматов. Это решение позволяет добавлять поддержку
новых форматов не затрагивая при этом внутреннею модель. В UGENE поддерживаются
чтение форматов биологических структур PDB и MMDB.

Эта модель была доработана, с целью иметь возможность выполнять трехмерные
преобразования структуры, и использована в работе.

\subsection{Подмножество макромолекулы BioStruct3DSubset}
Для построения структурного выравнивания важно наличие возможности указать
регион молекулярной цепи, на котором оно будет выполняться. BioStruct3D не
обеспечивает подобной функциональности, поэтому был добавлен тип
\class{BioStruct3DSubset}, представляющий из себя ссылку на подмножество
структуры (\lookat{fig:BioStruct3DSubset}).

\includefigure[scale=0.6]{Подмножество биологической
структуры}{BioStruct3DSubset}

В него входят следующие компоненты:
\begin{itemize}
    \item \textbf{BioStruct}  ссылка на полную структуру
    \item \textbf{chains} список ID молекулярных цепей
    \item \textbf{chainRegin} регион молекулярной цепи (для случая когда задана
одна цепь)
    \item \textbf{modelId} используемая модель
\end{itemize}

Также реализован элемент графического интерфейса, позволяющий пользователю
задавать регион.
\includefigure[width=0.6\linewidth]{Настройки структурного
выравнивания}{subset-editor}

\subsection{Интерфейс алгоритма выравнивания}
В результате анализа задачи, был выделен минимальный достаточный интерфейс
алгоритма парного структурного выравнивания
(\lookat{fig:StructuralAlignmentAlgorithm}).

\includefigure[scale=0.6]{Интерфейс алгоритма
выравнивания}{StructuralAlignmentAlgorithm}

Выравнивание выполняется с помощью вызова \method{align()}, он принимает на
входе ссылки на два описания структуры \class{BioStruct3DSubset} и возвращает в
результате экземпляр \class{StructuralAlignment}, содержащий рассчитанные
матрицу трансформации и RMSD.

Метод \method{validate()} позволяет быстро проверить входные данные на
соответствие ограничениям конкретного метода выравнивания. Это позволяет
предупредить пользователя о некорректных параметрах непосредственно на этапе их
ввода.

Реализации интерфейса \class{StructuralAlignmentAlgorithm} должны помещаться в
подключаемых модулях на базе интерфейса \class{Plugin}, предусмотренного в
UGENE. При загрузке подключаемый модуль регистрирует реализацию алгоритма в
специальном реестре, откуда он в последствии может быть получен. Таким образом,
возможно добавление поддержки новых алгоритмов.

\subsection{Интерфейс вычислительной задачи UGENE Task}
Каждый отдельный алгоритм или его независимая часть базируется на интерфейсе
\class{Task} (\lookat{fig:Task}) существующем в UGENE. Интерфейс \class{Task}
используется для описания некоторого независимого процесса или действия внутри
системы. Выполнением этих действий управляет менеджер задач UGENE. Такой подход
обеспечивает унификацию и, в ряде случаев, позволяет легко распараллелить
выполнение программы.

\includefigure[scale=0.6]{Интерфейс вычислительной задачи}{Task}

\subsection{Библиотека PTools}
Реализация самого алгоритма с нуля не имеет смысла, так как на данный момент
существует множество готовых библиотек и инструментов поддерживающих эти
функции. Причем, часть из них имеют открытый исходный код. Было решено выбрать
одну из таких библиотек, отвечающую следующим требованиям:

\begin{itemize}
    \item Актуальность и эффективность алгоритма
    \item Доступный исходный код и лицензия совместимая с GPLv2 \cite{gpl}
    \item Код на языках C/С++
    \item Переносимость: должны поддерживаться платформы Microsoft Windows,
Linux и Mac, как 32 так и 64 разрядные
    \item Гибкость и высокий потенциал развития
    \item Наличие документации
\end{itemize}

В итоге была выбрана библиотеке докинга PTools \cite{ptools}, в которой имеется
возможность наложения структур. Алгоритм, присутствующий в данной библиотеке,
позволяет строить наложения трехмерных структур, путем вычисления трехмерной
трансформации минимизирующей RMSD \cite{ptools-method}.

Для встраивания в UGENE код библиотеки был существенно переработан. Модули
разделены, оставлены только те из них которые непосредственно используются
алгоритмом выравнивания. Код адаптирован для библиотеки Qt и системы сборки
qmake. Участки кода, использующие нестандартные расширения GCC, переписаны,
чтобы обеспечить совместимость с компилятором MSVC.

Внутренняя модель данных библиотеки для представления макромолекулярной
структуры значительно отличается от модели существующей в UGENE, что делает
невозможным использование \class{BioStruct3D} и \class{BioStruct3DSubset}
напрямую. Для решения этой проблемы был разработан адаптер, который осуществляет
преобразование из одного внутреннего формата в другой. Следует отметить, что
конвертация требуется только для входных структур, так как рассчитанная в
результате работы выравнивания матрица трансформации может быть без проблем
применена к \class{BioStruct3D} напрямую.

Код библиотеки вместе с реализацией интерфейса
\class{StructuralAlignmentAlgorithm} размещен в подключаемом модуле
\module{PTools}, который, в свою очередь включен в дистрибутив UGENE.

\subsection{Визуализация выравнивания}
Визуализация структурного выравнивания реализована на базе ранее существовавшего
компонента UGENE \module{BioStruct3DView}. Этот инструмент предоставляет широкие
возможности визуализации биоструктур. Однако, изначально BioStruct3DView не
предназначен для визуализации выравниваний. Для этой задачи прежде всего
необходима возможность размещать несколько трехмерных структур в одной сцене, а
также гибко управлять параметрами отображения для каждой из них, чтобы избежать
загромождения итогового изображения.

\includefigure[width=0.6\linewidth]{Визуализация выравнивания молекул
тиоредоксина человека (3TRX) и чернобрюхой дрозофилы
(1XWC)}{alignment-3TRX-vs-1XWC}

Добавление этой функциональности повлекло за собой значительные изменения в
BioStruct3DView, так как этот модуль проектировался без учета озвученных
требований. Также добавлена новая цветовая схема, улучшающая визуализацию
выравнивания; улучшено управление списком отображаемых моделей и отображение
выделенных частей молекулы.

\subsection{Тестирование}
Добавленная функциональность была тщательно протестирована. С помощью
специального фреймворка, входящего в состав UGENE, были созданы автоматические
тесты для алгоритма выравнивания проверяющие корректность его работы и тесты
подтверждающие, что изменения в коде библиотеки PTools не влияют на логику её
работы и результаты. Кроме того, библиотека PTools изначально поддерживает
загрузку файлов формата PDB. Поэтому были добавлены тесты, позволяющие
убедиться, что результаты выравнивания не зависят от того какой загрузчик PDB
используется будь то встроенный в библиотеку, или загрузчик UGENE вместе с
последующим преобразованием из BioStruct3D во внутренний формат PTools. В
качестве тестовых данных использовались как специально сгенерированные наборы
данных с заранее известными проверенными результатами так и реальные данные из
базы данных PDB. Работа компонентов была успешно протестирована на ряде
платформ, таких как Microsoft Windows 7 64bit, Linux x86, x86\_64, Mac OS X.
\end{original}

%\newpage
%\section{Results}

% -------------------- Заключение -------------------------------------
\section{Conclusion}
\begin{original}
В данной работе ставилась цель разработать инструментарий позволяющий выполнять
построение и визуализацию структурного выравнивания. Эта цель была достигнута, и
были удовлетворены поставленные в разделе 4 требования, включая наглядный
пользовательский интерфейс и возможности для дальнейшего расширения
функциональности. Более того, были устранены недоработки существовавшего
инструмента визуализации BioStruct3DView.
\end{original}

For this work the goal was set to develop a toolkit for structural alignment
construction and visualization. This goal was achieved and the requirements
listed in sec. 4 were met, including graphical user interface as well as the
programming interface ready for further extension. Moreover, the existing
visualization tool BioStruct3DView was improved.

\begin{original}
Важной особенностью данной реализации является тесная интеграция со средствами
анализа UGENE сразу на нескольких уровнях биологической структуры (первичном,
вторичном и третичном).
\end{original}

Another important feature of this implementation is tight integration with the
UGENE multi-level analysis means (primary, secondary and tertiary).

\begin{original}
Наличие простого графического пользовательского интерфейса позволяет
исследователям биологам применять его в своей работе, не обладая при этом
специфичными навыками программирования.
\end{original}

The presence of a graphical user interface allows researchers biologists to
apply these tools in their work without possessing a specific programming
skills.

\begin{original}
Разработанные программные интерфейсы, <<опробованные>> путем реализации
конкретного метода выравнивания, делают возможным дальнейшее пополнение
библиотеки алгоритмов выравнивания и расширение функциональности визуализатора:
добавление специфичных для выравнивания стилей визуализации и цветовых схем,
поддержка различных форматов хранения результатов выравнивания.
\end{original}

The developed program interfaces tested by implementing particular method of
alignment make it possible to extend the library of alignment algorithms and to add
visual representations (i.e. styles and color scheme) to visualizer. This will
allow researchers to choose the most suitable tools for their application or even to
develop their own.

\begin{original}
Вся добавленная функциональность была тщательно протестирована. Созданная
тестовая база не только подтверждает, что разработанные компоненты работают
корректно, но и позволяет избежать ошибок при внесении улучшений.
\end{original}

All implemented modules were carefully tested. Created test base can not only
confirm that components work correctly, but also will ease modifications in
future by validating incoming changes.

\begin{original}
Заметным результатом проделанной работы является тот факт, что инструмент
построения выравнивания и визуализатор как часть UGENE, вошли ив дистрибутивы
открытых операционных систем Ubuntu Linux и Fedora Core.
\end{original}

A notable result of this work is the fact that the structural alignment tool and
renderer as part UGENE are included in open source operating systems
distributions such as Ubuntu Linux and Fedora Core. Usage of these distributions
are widely spread in the scientific environment.

\begin{original}
На этом развитие разработанного инструментария не прекращается, планируется
продолжить работу над ним, а также над реализацией других средств структурного
анализа, необходимых молекулярным биологам в реальных исследованиях.
\end{original}

At this point evolution of named tools doesn't stop. Existing functionality
improvement as well as implementing new structural analysis means enabling
actual, convenient instruments for researchers-biologists will be continued in
further work.


\begin{original}
В рамках дипломной работы был разработан, реализован и протестирован инструмент
для построения структурного выравнивания. Использование разработанного
инструментария в совокупности с инструментами UGENE в реальных исследованиях
позволит исследователям-биологам упростить и ускорить процесс изучения
биологических структур, а также решение других актуальных задач в данной
области. При этом сам процесс анализа является интерактивным и легко управляемым
благодаря наличию дружелюбного графического пользовательского интерфейса и
возможности трехмерной визуализации результатов.

Открытая модель разработки проекта Unipro UGENE, в рамках которого выполнялась
данная работа, делает для исследователя возможным расширять функциональность
инструментария за счет своих собственных решений. И поэтому, наряду с
реализованным алгоритмом выравнивания важным результатом работы также является
программный интерфейс, спроектированный с учетом возможности дальнейшего
расширения и добавления новых методов анализа в будущем.

В ходе выполнения работы был получен важный разработки масштабных наукоемких
программных продуктов. Ключевыми этапами в накоплении опыта было изучение и
применение на практике методов биоинформатики, изучение и использование
фреймворка Qt, приобщение к открытой модели разработки программного обеспечения,
знакомство с принципами мультиплатформенного и параллельного программирования,
изучение промышленных методов, используемых в разработке и тестировании
программного обеспечения.

Проект UGENE, в рамках которого выполнялась работа, на данный момент продолжает
свое развитие. Добавляются новые методы анализа, создаются специализированные
платформо-ориентированные оптимизации алгоритмов (GPGPU, кластерные и облачные
решения).
Уже сейчас вычислительная платформа UGENE, вместе с разработанным в ее контексте
инструментарием входит в состав дистрибутивов двух открытых операционных систем:
Ubuntu Linux и Fedora Core.
Исследования, проведенные в рамках данной работы, имеют высокий потенциал для
развития и могут служить хорошей базой для дальнейших разработок в области
вычислительной структурной биологии.

\end{original}


% -------------------- Список литературы-------------------------------
\begin{flushleft}
    \nocite{algolist-cgl}           % http://www.cgl.ucsf.edu/home/meng/grpmt/structalign.html
    \nocite{algolist-protopedia}    % http://proteopedia.org/wiki/index.php/Structural\_alignment\_tools

    \bibliography{biblio/diploma}
\end{flushleft}


\end{document}
